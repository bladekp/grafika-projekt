\documentclass[a4paper,11pt,notitlepage]{article}
\usepackage[utf8]{inputenc}
\usepackage[T1]{fontenc}
\usepackage[polish]{babel}
\usepackage[MeX]{polski}
\selectlanguage{polish}
\hyphenation{FreeBSD}
\author{Piotr Błądek}
\usepackage{geometry}
\newgeometry{tmargin=3cm, bmargin=3cm, lmargin=3cm, rmargin=3cm}
\title{Grafika komputerowa projekt \\ Sprawozdanie z projektu eliminacji elementów zasłoniętych}
\date{\today}
\linespread{1.3}
\usepackage{indentfirst}
\begin{document}
\maketitle

\section{Sposób zrealizowania}

Zadanie zostało zrealizowane w postaci funkcji dopisanej do programu wirtualnej kamery z projektu pierwszego. W każdej iteracji wykonuje się liczenie i eliminowanie elementów zasłoniętych ze sceny. Do zadania został wykorzystany algorytm malarski z liczeniem średniej odległości wieloboku od obserwatora. 

\section{Zgodność z założeniami}

Funkcja spełnia założenia jakie jej postawiłem, elementy renderują się w dobrej kolejności i wyglądaja całkiem realistycznie kiedy się przy nich poruszamy. Występują problemy kiedy na przykład znajdujemy się blisko dwóch ścian, ale to normalne zjawisko przy tak prostym algorytmie, można je optymalizować ale wtedy chyba łatwiej i przyjemniej byłoby po prostu zmienić algorytm na bardziej profesjonalny.

\section{Wnioski}

Do spełnienia wymogów tego zadania wystarczyła implementacja algorytmu malarskiego, jednak nie stosowałbym go w bardziej poważnym projekcie. Jego zaletami na pewną są prosta konstrukcja algorytmu i trywialne sortowanie, ale ma sporo wad takich jak brak wsparcia dla cyklicznego zasłaniania lub przecinania wieloboków. 

\end{document}
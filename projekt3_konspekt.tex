\documentclass[a4paper,11pt,notitlepage]{article}
\usepackage[utf8]{inputenc}
\usepackage[T1]{fontenc}
\usepackage[polish]{babel}
\usepackage[MeX]{polski}
\selectlanguage{polish}
\hyphenation{FreeBSD}
\author{Piotr Błądek}
\usepackage{geometry}
\newgeometry{tmargin=3cm, bmargin=3cm, lmargin=3cm, rmargin=3cm}
\title{Grafika komputerowa projekt \\ Konspekt projektu oświetlenia kuli}
\date{\today}
\linespread{1.3}
\usepackage{indentfirst}
\begin{document}
\maketitle

\section{Opis}

Zadanie polega na zaimplementowaniu algorytmu który da nam wrażenie że gdzieś na scenie jest źródło światła. Najogólnej rzecz ujmując chodzi o to aby tak pokolorować wielobok tai żeby ich kolory odzwierciedlały warunki panujące na scenie (padające światło).

\section{Sposób realizacji}

Do wykonania zadania dopiszę do projektu wirtualnej kamery funkcję która w zależności od położenia źródła światła odpowiednio pokoloruje wieloboki znajdujące się na scenie. Do wyliczenia jak pokolorować wieloboki skorzystam z gotowego modelu empirycznego Phonga. Dodatkowo w programie zostanie dodana funkcjonalność dostosowywania parametrów takich jak:

\begin{enumerate}
 \item Klawisz F1 - zmniejszenie stopnia wypolerowania obiektów.
 \item Klawisz F2 - zwiększenie stopnia wypolerowania obiektów.
 \item Klawisz F3 - zmniejszenie współczynnika światła rozproszonego.
 \item Klawisz F4 - zwiększenie współczynnika światła rozproszonego.
 \item Klawisz F5 - zmniejszenie współczynnika światła otoczenia.
 \item Klawisz F6 - zwiększenie współczynnika światła otoczenia.
\end{enumerate}

Oraz do zmiany kolorów obiektów na scenie kolejno zmniejszanie czerwonego klawiszem F7, zwiększanie czerwonego klawiszem F8, zmniejszanie zielonego klawiszem F9 ... zwiększanie niebieskiego klawiszem F12.

\end{document}
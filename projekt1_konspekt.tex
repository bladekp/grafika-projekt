\documentclass[a4paper,11pt,notitlepage]{article}
\usepackage[utf8]{inputenc}
\usepackage[T1]{fontenc}
\usepackage[polish]{babel}
\usepackage[MeX]{polski}
\selectlanguage{polish}
\hyphenation{FreeBSD}
\author{Piotr Błądek}
\usepackage{geometry}
\newgeometry{tmargin=3cm, bmargin=3cm, lmargin=3cm, rmargin=3cm}
\title{Grafika komputerowa projekt \\ Konspekt projektu wirtualnej kamery}
\date{\today}
\linespread{1.3}
\usepackage{indentfirst}
\begin{document}
\maketitle

\section{Opis}

Projekt zakłada stworzenie programu do wyświetlania grafiki 3D na ekranie monitora. Patrzymy przez tzw. wirtualną kamerę w pewnym kierunku. Naszym zadaniem jest zrzutowanie obiektów znajdujących się w przestrzeni trójwymiarowej w kierunku wskazywanego przez obiektyw naszej kamery na płaszczyznę rzutni tak aby jak najwierniej odwzorować kształty rzeczywiste obiektów. Wirtualną kamerą możemy poruszać w różnych osiach i płaszczyznach:

\begin{enumerate}
\item Do przodu i do tyłu.
\item W lewo i w prawo.
\item Do góry i w dół.
\item Obracać dookoła osi kamery.
\item Robić zoom przestrzeni.
\end{enumerate}

\section{Sposób realizacji}

Do rysowania użyję bibliotek Swing i AWT języka Java. Główną klasą będzie Ekran gdzie będę rysował obiekty. Napiszę konwerter który z wieloboku 3D wyliczy mi wielobok 2D który następnie będę mógł narysować na ekranie. Napiszę klasy pomocnicze tj. Wektor (do przechowywania wektora znormalizowanego i liczenia ilorazu wektorów), Punkt (do przechowywania współrzednych punktu który będę rzutował), Rzutnia (do przechowywania parametrów rzutni). Panel z rysunkiem będzie odświeżany z największą możliwą częstotliwością.

\end{document}
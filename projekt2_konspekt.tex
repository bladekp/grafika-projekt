\documentclass[a4paper,11pt,notitlepage]{article}
\usepackage[utf8]{inputenc}
\usepackage[T1]{fontenc}
\usepackage[polish]{babel}
\usepackage[MeX]{polski}
\selectlanguage{polish}
\hyphenation{FreeBSD}
\author{Piotr Błądek}
\usepackage{geometry}
\newgeometry{tmargin=3cm, bmargin=3cm, lmargin=3cm, rmargin=3cm}
\title{Grafika komputerowa projekt \\ Konspekt projektu eliminacji elementów zasłoniętych}
\date{\today}
\linespread{1.3}
\usepackage{indentfirst}
\begin{document}
\maketitle

\section{Opis}
\noindent
Zadanie polega na napisaniu funkcji która dla programu z projektu wirtualnej kamery wyliczy kolejność rysowania poszczególnych wieloboków tak żeby obserwatorowi pokazywały się one w naturalnej kolejności (od tych najdalej położonych do tych najbliżej obserwatora). W ten sposób z punktu widzenia operatora kamery obraz będzie wyglądał naturalnie.

\section{Sposób realizacji}
\noindent
Do realizacji tego zadania skorzystam z algorytmu malarskiego, jest to najprostszy w implementacji algorytm który jest wystarczający dla moich celów. W funkcji która będzie implementowała algorytm będę liczył średnią odległość wieloboku od użytkownika i na tej podstawie rysował kolejno od najdalej od obserwatora położonego wieloboku do tego który będzie położony najbliżej. W razie problemów w trakcie tworzenia sceny pomyślę nad jej optymalizacjami. \newline
Funkcja zostanie zaimplementowana w projekcie wirtualnej kamery, przez co jej działanie będzie można łatwo przetestować w gotowej scenie.

\end{document}